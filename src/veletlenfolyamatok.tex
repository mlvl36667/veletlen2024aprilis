\documentclass{article}
\usepackage{amsmath}
\usepackage{enumitem}
\begin{document}

\title{Véletlen folyamatok}
\author{Ladóczki Bence}
\date{\today}

\maketitle

\section{Bevezetés}

Véletlen (sztochasztikus) folyamatnak nevezzük azon empirikus folyamatok matematikai absztrakcióját, amit valószínűségi törvények vezérelnek. A jelen kéziratban a véletlen folyamatok elméletének csupán egy részhalmazával foglalkozunk. A megértéshez szükség van alapvető valószínűség számítási előismeretekre. 

A valószínűség számítás a mértékelmélet egy speciális tárgyalásmódja, annak egy alkalmazott ága. Elviekben van arra is lehetőség, hogy a minta-sorokat és az azokon értelmezett függvényeket a valószínűségi számítástól függetlenül tárgyaljuk és hagyományos sorokkal valamint függvényekkel foglalkozunk és a valószínűség számításra mint az eloszlások tanulmányozására tekintsünk. Egy időben a matematika csak így tudta például a nagy számok erős törvényét teljesen egzakt módon értelmezni. Már azonban tudjuk, hogy erre nincs szükség és ez sok helyen túlságosan lerövidítené az anyagot és elferdítené a valóságot. 

A valószínűség számítást az teszi egyedivé a matematikai területek között, hogy a a legegyszerűbb törvényszerűségei is meglehetősen bonyolult matematikai törvényszerűségekből következnek, viszont a bonyolultabb eredményeket is ettől függetlenül ismerik és alkalmazzák a gyakorlatban (gondoljunk például a statisztikusokra). A valószínűség számítás valóban egzakt tárgyalásmódja ellen gyakran felhozzák azt az érvet, miszerint nincs szükség ilyen szinten belemenni a részletekbe. A könyv végére mindenki hozza meg magának a saját ítéletét.

\begin{flushright}
2024. április 14. \\ Budapest
\end{flushright}

\section{Bevezetés és Valószínűség Elméleti Alapok}

A jelen kéziratban bonyolult matematikai módszerek tárházát használjuk, ettől függetlenül azonban mindig a valószínűség számítás nyelvének segítségével mondjuk majd ki az eredményeket. Remélem, hogy azon specialisták számára akik tisztában vannak vele, hogy hogyan kell mondjuk feltételes valószínűséget vagy várható értéket számolni nem fog problémát okozni a megértés. Ebben a fejezetben a legfontosabb matematikai alapokkal ismerkedünk meg. 

Amikor magasabb szintű valószínűség számítással foglalkozó könyvet szeretnénk megírni --~mivel a valószínűség számítás a mértékelmélet egy ága~-- két választás előtt áll a szerző. Vagy hagy egy fejezetet a mértékelméletnek, vagy feltételezi, hogy az olvasó a rendelkezésre álló egyéb mértékelméleti szakkönyvekből (előadásokból) már elsajátította a szükséges előismereteket. Praktikussági okokra hivatkozván a második opciót választom, és azt tartom szem előtt, hogy itt egy olyan kézirat készül, amit a mértékelmélet alapos ismerete nélkül is lehet abszolválni. A valószínűség számítás alapvető összefüggéseit, az abban használt módszereket azonban ismertnek feltételezem.

\subsection{Milyen Térben Dolgozunk?}

Amióta a matematikusok a valószínűség számításra egzakt tudományra tekintenek az olyan fogalmakra mint például a \textit{megjelenés}, \textit{esemény}, \textit{urna}, \textit{kocka} és hasonló nincs többé szükség, amikor valószínűség számításról beszélünk. Ettől függetlenül, ha továbbra is használjuk ezeket, akkor az intuíció a segítségünkre lesz az elkövetkezendő diszkussziók során és analitikus módszereket dolgozhatunk majd ki, valamint új kutatási irányok nyílhatnak meg. Főként emiatt a mai napig gyakran a szóhasználat megmarad még akkor is ha tisztán elméleti valószínűség számításról beszélnek a matematikusok is.

A valószínűség számítás alkalmazása során a \textit{valószínűséget} események bekövetkezésével azonosítják. Olyan kérdéseket szoktunk feltenni, hogy például mi annak a valószínűsége, hogy $5$-öt dobunk egy dobókockával? Vagy ha egy részecskre Brown mozgást végez, akkor mi lesz az $x$ irányú elmozdulás centiméterben kifejezett mérőszámának valószínűségi eloszlása $t$ idő múlva ($x(t)$)? Az ilyen eseményekhez rendeljük a valószínűséget. Ahhoz például, hogy $5$-öt dobunk egy dobókockával az $\frac{1}{6}$ valószínűséget rendeljük hozzá. Ahhoz, hogy páros számot dobunk, az $\frac{1}{2}$ valószínűséget rendeljük hozzá. Amikor Brown mozgásról beszélünk, akkor az $x(3) > 7$ eseményhez a~\ref{sec:fuggetlennovekmenyufolyamatok}.\ fejezetben ismertetet szabályok szerint kell valószínűséget rendelni. A kocka esetében a következő módon lehet a matematika nyelvén megfogalmazni a jelenséget. Az összes lehetséges kimenetelhez hozzárendeljük az egész számokat: $1 \ldots 6$ és ezekhez a számokhoz az $\frac{1}{6}$ valószínűséget rendeljük hozzá. Ezek tetszőleges osztályához pedig az $\frac{n}{6}$ valószínűséget rendeljük hozzá. Az például hogy páros számot kapunk csak annyit jelent, hogy $n = 3$-ra kiértékeltük a formulát és $\frac{3}{6} = \frac{1}{2}$ kapunk valószínűségnek. Ezt a gondolatmenetet elvégezni a Brown mozgás esetére kicsivel bonyolultabb és emiatt ezt most nem tesszük meg, viszont megjegyzem itt is csupán arról van szó, hogy események matematikai absztrakicójához számokat kell rendelni. Ezentúl az események matematikai absztrakciójának egy \textit{ponthalmazt} választunk. 

A valószínűség számítás elmélete nem más mind a különböző terek mérhetőségi tulajdonságainak vizsgálata, valamint az ezeken a tereken definiált mérhető függvények analízise. A valószínűség számítás alkalmazási területeire való tekintettel hívhatjuk ezeket a tereket \textit{mintavételi} tereknek, a mérhető halmazokat pedig \textit{eseményeknek}, ez a megközelítés azonban nem mindig állja meg a helyét. 

Most pedig egy következetes matematikai tárgyalásmód kerül bevezetésre. Feltesszük tehát, hogy egy $\Omega$ alaptérrel van dolgunk és hogy ezen a téren ponthalmazok vannak. Ezeket a halmazokat hívjuk majd \textit{mérhető halmazoknak}. Feltesszük, hogy a mérhető halmazok osztálya Borel mezőt alkot, továbbá azt is megköveteljük, hogy legyen egy $\mathbf{P}\{\cdot\}$ függvény, amely minden mérhető halmazra definiálva van. Ezt hívjuk \textit{valószínűségi mértéknek}, ami azt jelenti, hogy $\mathbf{P}\{\cdot\}$ értelmezhető minden mérhető halmazra, nem negatív és tökéletesen additív a teljes téren pedig $1$-et vesz fel. A $\mathbf{P}\{\Lambda\}$ számot $\Lambda$ \textit{valószínűségének} vagy \textit{mértékének} nevezzük. Egy $\omega$ függvény mérhetőségét hasonló módon kell megadni a mérhető halmazokkal valamint a valószínűségi mérték szerinti integrállal kifejezve. Ez az egy mérhető $\phi$ függvény mérhető $\Lambda$ halmazok feletti integrálját jelenti:
\begin{equation}
\int_{\lambda} \phi(\omega) d \mathrm{\textbf{p}} \quad \mathrm{vagy} \int_{\lambda} \phi d \mathrm{\textbf{p}}
\end{equation}
Egy tulajdonságra azt mondjuk hogy \textit{majdnem mindenhol} igaz, ha $\Omega$ minden pontjára igaz kiveéve a $0$ valószínűségű pontokat. Erre azt is mondhatjuk, hogy majdnem minden $\omega$ pontban igaz, vagy hogy $1$ valószínűséggel igaz. 

Nézzük például azt az esetet, amikor egyszer dobunk a dobókockával. Ebben az esetben az $\Omega$ tér az $1 \ldots 6$ számokból áll. Egy $j$ pontot azzal az eseménnyel azonosítjuk, amikor $j$-t dobunk a kockával. Ebben az esetben minden $\Omega$ halmaz mérhető és minden halmazhoz a benne lévő pontok számának az $\frac{1}{6}$-át rendeljük hozzá. Így joggal beszélhetünk egy mintavételi térről, hiszen a tér pontjai és a lehetésges események között egyértelmű megfeleltetést tudtunk felállítani. Most tekintsük ugyanezen kísérlet egy más matematikai modelljét. Ezen alkalommal az $\Omega$ térbe vegyük bele az összes valós számot és és a kockával való dobásból származó eseményekhez rendeljük azokat a számokat a térből, mint az előbb. Más szabályt ne állítsunk fel. Ebben az esetben is minden $\Omega$ ponthalmaz mérhető és minden halmaz mértéke $\frac{1}{6}$-a annak, hogy hány egész szám kerül bele az $1 \ldots 6$ intervallumból. A matematikai modell, amit így kapunk majdnem teljesen ugyan az, mint az előző esetben, azonban a mintavételi tér elnevezés kicsit kevésbé praktikus, hiszen így már nehezebb megfeleltetni minden pontot egy valós eseménnyel. Végezetül tekintsünk egy harmadik felállást, amikor is az $\Omega$ tér az \([1,7)\) félig nyílt intervallumon található összes számból áll. Most feltessük meg a \([$j$,$j+1$)\) intervallumot annak az eseménynek, amikor $j$-t dobtunk. Ebben az esetben a mérhető pontok halmazán intervallumokat (\([$j$,$j+1$)\)), vagy ezek unióját értjük. És bármely mérhető halmaz mértéke megegyezik az intervallumok darabszámának $\frac{1}{6}$-ával. Ezt a modellt nyilván ugyanolyan jól lehet használni mint az előző kettőt, azonban a mintatér elnevezés itt sem állja meg a helyét. Nyilván nem utasíthatjuk vissza ezt az okfejtést, mondván hogy a modell túl bonyolult. A feleslegesen bonyolult modellek még a gyakorlati alkalmazásokban sem söpörhetőek le az asztalról. A matematikai modell felállításának módja attól függ, hogy milyen paramétereket veszünk alapul. Az első modellben konkrétan a dobások kimenetelét vettük alapul. Azonban ha a dobókockára úgy tekintünk, mint egy fizikai objektum, ami az descartesi térben mozog valamilyen sebességgel, akkor a $2$ oldalának hely és sebesség vektorai egyértelmű meghatározzák a dobás kimenetelét ($12$ paraméter). Az $\Omega$ térnek tehát természetes módon választhatnánk a $12$ darab kiindulási hely és sebesség komponenseket. Az $\Omega$ tér egy pontja tehát egyértelműen meg fogja határozni a dobás kimenetelét. Jelöljük $\Lambda_j$-vel azokat a pontokat a térben, amik végül majd $j$ dobást eredményeznek. Az előző gondolatmenet alapján minden $\Lambda_j$ halmazhoz $\frac{1}{6}$ valószínűséget kell rendelnünk. Csupán ennyi tudásra van szükségünk az $\Omega$ térről, ha csak a dobás kimenetelére vagyunk kíváncsiak. Ez a modell nagyon hasonlít az előbbiekben ismertetett harmadik modellre. Arról van szó tehát, hogy mind az elméleti, mind a gyakorlati tárgyalásmódok során, megválaszthatjuk az $\Omega$ teret olyan kritériumok alapján, ami egyszerűsíti a kezelését (hogy lehessen mondjuk mintavételi térnek nevezni), azonban ez a kritérium pont az azt követő levezetésekben fog elveszni, amikor is valószínűségi eloszlásokat hozunk ki a kezdeti felállásból. Pontosan emiatt az $\Omega$ térrel kapcsolatban nem kötünk ki semmi olyat, ami a $\mathbf{P}\{\Lambda\}$ halmazfüggvény létezéséből implicit módon ne következne. A $\mathbf{P}\{\Lambda\}$ létezéséből ugyan következik, hogy az $\Omega$ tér nem lehet üres, de ennél többet nem várunk el az $\Omega$ tértől. A továbbiakban az $\Omega$ tér alatt egy absztrakt teret értünk. Amikor konkretizáljuk a teret, akkor az $\Omega$ térnek az $0 \leq x \leq 1$ intervallumot választjuk. Ez lehet továbbá az összes véges vagy végtelen valós értékű függvénye a $t$ változónak, ahol $-\infty < t < \infty$, vagy a véges sík is akár. Ami valamelyest a valószínűség számítással kapcsolatos, az az, hogy a $\mathbf{P}\{\Lambda\}$ függvénytől megköveteljük a normalizálási feltételt, avagy $\mathbf{P}\{\Lambda\} = 1$. Ha elrugaszkodunk a valószínűség számítástól, akkor nem is mindig kell megkövetelni, hogy a halmazfüggvények véges értékűek legyenek, ha pedig végesek, akkor arra sincs feltétlenül szükség, hogy a $\mathbf{P}\{\Lambda\} = 1$ feltétel teljesüljön. 

Foglalkozzunk a továbbiakban megint a dobókockával való dobással, hiszen ezen a például keresztül érthetjük meg miért is olyan praktikus az alaptérként szorzatteret választani. Ha végtelenszer eldobjuk a dobókockát a következő módon könnyen definiálhatunk alapteret. Az $\Omega$ térben legyen minden $\omega$ pont egy sorozat: $\xi_1, \xi_2,\ldots$, ahol minden $\xi_i$ egy egész lehet az $1 \ldots 6$ intervallumból. Az $\Omega$ tér egy pontja tehát lehet egy olyan sorozat, ami akkor valósul meg, maikor a dobókockát végtelenszer eldobjuk. Ebben az esetben azonosíthatjuk az összes olyan sor osztályát, ami mondjuk $j$-vel kezdődik. Ezen sorok osztályához az $\frac{1}{6}$ valószínűséget rendelhetjük hozzá és így az egyszeri dobókocka dobás egy másik matematikai modelljét kapjuk meg. Az ilyen jellegű megválasztásban az a szerencsés, hogy ezt a modellt ki lehet bővíteni tetszőleges számú dobásra ha az osztályokat a $j_1,\ldots,j_n$-el kezdődő sorokból képezzük és ezekhez rendeljük azt az eseményt, miszerint az első $n$ dobás során pontosan $j_1,\ldots,j_n$-et kapunk. Ezen sorok osztályához ($\Omega$ halmaz) az $(\frac{1}{6})^{n}$ valószínűséget rendelhetjük hozzá. Valamennyire igazuk lehet azokonak akik azt, mondják, hogy a végtelen darabszámú kísérletet a gyakorlatban sosem végzünk el, azonban a gyakran még a legegyszerűbb esetben sem tudjuk kikerülni a végtelen sorok tárgyalását. Például ha analizálni szeretnénk azt az eseményt, amikor valami először következik be, mondjuk először hatot dobunk a kockával, akkor ezt csak végtelen sorokon értelmezett mintatéren tudjuk megtenni, hiszen azt a számot, ami ahhoz kell, hogy valami megtörténjen nem tudjuk alulról becsülni mielőtt elvégeznénk a kísérletet. Ez a szám az $\Omega$ téren egy nem korlátolt függvény. Az előző példában az $\Omega$ tér szorzattér és végtelen sok faktortere van, ezek pedig mind hat pontot tartalmaznak. 

Az $\Omega$ téren értelmezhetünk feltételeket, ezeket $C$-vel jelöljük és az $\{C\}$ jelölést használjuk majd azokra a pontokra, amik kielégíti az $\Omega$ téren a $C$ feltételt. Például ha $X$ lineáris halmaz és ha $x$ egy $\omega$ függvény, akkor $\{x(\omega) \in X\}$ egy olyan $\omega$ halmaz, amelyen $x(\omega)$ egy szám az $X$ halmazból.

Még nem tettük fel, hogy az alap valószínűségi mérték teljes. Arról van szó, hogy ha $\Lambda_0$ mérhető és mértéke $0$ és ha $\Lambda$ $\Lambda_0$ részhalmaza, attól még nem kell, hogy $\Lambda$ mérhető legyen. Ha ezt le szeretnénk fordítani a valószínűség számítás nyelvére, azt mondhatjuk, hogy ha van két eseményünk, $\Lambda$ és $\Lambda_0$ és $\Lambda$ bekövetkeztéből következik $\Lambda_0$, akkor:
\begin{equation}
\mathbf{P}\{\Lambda_0\} = 0 \quad ,
\end{equation}
és ettől függetlenül $\Lambda$ valószínűsége nem biztos, hogy értelmezett. Nyilván, ha $\mathbf{P}\{\Lambda\}$ értelmezett, akkor az $0$. Egy matematikus számára ez a tulajdonság zavaró lehet ha meg szeretné magának tartani a valószínűség intuitív értelmezését. Itt tehát két választás előtt állunk, vagy módosítjuk intuíciónkat vagy megváltoztatjuk a matematikát. A választás nem túl fontos, azonban ha nagyon ragaszkodunk az intuíciónkhoz, akkor könnyen átdefiniálhatjuk fogalmainkat. Valójában a valószínűségi mértéket könnyedén teljessé lehet tenni ha az $\omega$ halamazok osztályát némileg kibővítjük. Ettől függetlenül, most azonban mégis tegyük inkább fel, hogy a valószínűségi mérték, amivel ezentúl dolgozni fogunk teljes.

\subsection{Valószínűségi Változók és Eloszlások}\label{ss:vveelo}

Egy valós $x$ függvényt egy $\omega$ ponttéren valós \textit{valószínűségi változónak} nevezünk ha az $\omega$ halmazokon létezik valószínűségi mérték és egy tetszőleges $\lambda$ valós számhoz az $x(\omega) \leq \lambda$ egyenlőtlenség bekorlátozza az $\omega$ halmazt és így is mérhető halmazt kapunk. Ebből következő módon: 
\begin{equation}
F(\lambda) = \mathbf{P}\{x(\omega) \leq \lambda\}
	\label{eq:distributionfunction}
\end{equation}
létezik tetszőleges valós $\lambda$-ra. A matematika nyelvét használva ezt úgy is mondhatjuk, hogy egy (valós) valószínűségi változó tulajdonképp nem más, mint egy (valós) mérhető függvény. Egy komplex valószínűségi változó pedig egy $\omega$ függvény, aminek a valós és a képzetes részei mérhetők. Mostantól, ha mást nem állítunk, mindig fel fogjuk tenni, hogy ha esetleg több valószínűségi változóról lenne szó, akkor a valószínűségi változókat mindig ugyanazon $\omega$ téren fogjuk értelmezni. A matematikusok egy része már a mértékelmélet megjelenése előtt is foglalkozott valószínűségi változókkal, viszont akkor még nem volt tudatosítva hogy tulajdonképp mérhető függvények tulajdonságait vizsgálják. Az akkori tisztán valószínűség számítási nyelvezetet át lehet fordítani a mai mértékelméleti nyelvezetre. Ezek mellett meg fogjuk tartani a klasszikus valószínűség számítási terminusokat is, hiszen az alkalmazások során ez segítség lehet. Az analízisben gyakran praktikus lehet ugyanazt az a jelölést használni a függvényre, valamint a felvett értékére az értelmezési tartományában és ez a kétértelműség gyakran ki is fizetődik. Sok esetben azonban ennél pontosabban kell eljárnunk. A függvényeket általában egy betűvel fogjuk jelölni és más jelölést fogunk alkalmazni a függévény értékére az adott pontban. Így tehát az $x(\omega)$ alatt az $x$ függvény értékét értjük az $\omega$ pontban. Gyakran, amikor célszerű, akkor az $x$ függvényt az $x(\cdot)$ jellel jelöljük majd. 

Az (\ref{eq:distributionfunction})-ben definiált $F$ függvényt az $x$ valószínűségi változó \textit{eloszlásfüggvényének} hívjuk. Ez a függvény monoton, nem csökkenő és jobbról folytonos és igazak rá a következők:
\begin{equation}
	\lim_{\lambda \to -\infty} F(\lambda) = 0, \quad \lim_{\lambda \to \infty} F(\lambda) = 1 .
\end{equation}
Bármely olyan függvényt, ami eleget tesz ezen feltételeknek eloszlásfüggvényének fogunk hívni. Az eloszlásfüggvény eloszlást definiál és ez $A$ halmazok valószínűségi mértéke lesz:
\begin{equation}
 \int_{A} dF(\lambda) .
\end{equation}
Ez egy hagyományos Lebesgue-Stieltjes mérték, amit $F$ definiál. Ha pedig $F$ eloszlásfüggvény és valamilyen Lebesgue mérhető függvényre teljesül a következő:
\begin{equation}
	F(\lambda) = \int^{\infty}_{-\infty} f(\mu) d\mu, -\infty < \lambda < \infty,
	\label{eq:density}
\end{equation}
akkor $f$-et az $F$-hez tartozó sűrűségfüggvénynek hívjuk. Majdnem minden $\lambda$-ra a Lebesgue mértékben teljesül az is, hogy $F'(\lambda) = f(\lambda)$. Amikor tehát azt mondjuk, hogy $F$-hez tartozik sűrűségfüggvény, akkor ez alatt azt értjük, hogy létezik egy $f$ függvény, ami kielégíti (\ref{eq:density})-t. Az $f$ függvény tehát akkor létezik, ha $F$ abszolút folytonos. 

\textit{Többváltozós eloszlásfüggvénynek} hívjuk a következő $x_i,\ldots,x_n$ valószínűségi változókra értelmezett függvényt:
\begin{equation}
	F(\lambda_1,\ldots,\lambda_n)=\mathbf{P}\{x_j(\omega) \leq \lambda_j, j = 1,\ldots,n\}.
	\label{eq:density2}
\end{equation}
A (\ref{eq:density2}) által definiált függvény minden változójában monoton, nem csökkenő és jobbról folytonos. Igazak rá továbbá a következők is:
\[
\lim_{\lambda_j \to -\infty} F(\lambda_1,\ldots,\lambda_n) = 0, \quad j = 1,\ldots,n
\]
\[
\lim_{\lambda_1,\ldots,\lambda_n \to \infty} F(\lambda_1,\ldots,\lambda_n) = 1.
\]
Továbbá $\lambda_j \leq \mu_j, j = 1,\ldots,n$-re teljesül az is, hogy:
\begin{align*}
    &F(\mu_1,\ldots,\mu_n) - \sum^{n}_{j=1} F(\mu_1,\ldots,\mu_{j-1},\lambda_j,\mu_{j+1},\ldots,\mu_{n}) \\
    &+ \sum^{n}_{\substack{j,k=1 \\ j < k}} F(\mu_1,\ldots,\mu_{j-1},\lambda_j,\mu_{j+1},\ldots,\mu_{k-1},\lambda_k,\mu_{k+1},\ldots,\mu_{n}) \\
    &- \ldots + (-1)^{n} F(\lambda_1,\ldots,\lambda_n) \geq 0.
\end{align*}
Az egyenlet bal oldalán lévő mennyiség a következő valószínűséget fejezi ki $F$ segítségével:
\[
 \mathbf{P}\{\mu_{j} \leq x_j(\omega) \leq \lambda_j, j = 1,\ldots,n\}.
\]
Minden olyan $F$ függévnyt, ami eleget tesz ezen feltételeknek $n$-változós eloszlásfüggvénynek fogunk hívni. Egy ilyen függvény $n$ dimenzióban Lebesgue-Stieltjes mértéket definiál:
\[
	\int\limits_{\substack{A}} \ldots \int\limits_{\substack{A}} d_{\lambda_1,\ldots,\lambda_n} F(\lambda_1,\ldots,\lambda_n).
\]
Egy Lebesgue mérhető és integrálható $f$ függvényt pedig hasonló módon sűrűségfüggvénynek hívunk (minden $\lambda_1,\ldots,\lambda_n$-re) ha igaz rá a következő:
\begin{equation}
	F(\lambda_1,\ldots,\lambda_n) = \int^{\lambda_1}_{-\infty} \ldots \int^{\lambda_n}_{-\infty} f(\mu_1,\ldots,\mu_{n})d\mu_1,\ldots,d\mu_{n}
\end{equation}
Ha minden lineáris Borel halmazra ($X_1,\ldots,X_n$) teljesül, hogy:
\begin{equation}
	\mathbf{P}\{x_j(\omega) \in X_j, j = 1,\ldots,n \} = \prod_{j=1}^{n} \mathbf{P}\{x_j(\omega) \in X_j\},
	\label{eq:kfvv}
\end{equation}
akkor az $x_1,\ldots,x_n$ valószínűségi változókat \textit{kölcsönösen függetlennek} nevezzük. Ezt a feltételt meg lehet adni nyílt $X_j$ halmazokra is, intervallumokra is és akár olyan félig végtelen intervallumokra, mint például ($-\infty,b_j$]. A definíciókból következik, hogy csak azok a valószínűségi változók függetlenek, amelyek közös többváltozós eloszlása az egyes eloszlásfüggvények szorzataként áll elő. 

Ez előző bekezdésben egy valószínűségi változó helyett $x_j$-vel akár jelölhetünk $m_j$ darab valószínűségi változót ($x_{j1},\ldots,x_{jm_j}$) és ekkor ha $X_j$ ennek megfelelően egy $m_j$ dimenziós halmaz, akkor az előzőekben tárgyalt elmélet segítségével kölcsönös függetlenséget tudunk definiálni véges számú valószínűségi változó aggregátumra, amikben véges számú valószínűségi változó van. Ha valamelyik aggregátumban mégis végtelen sok valószínűségi változó lenne, akkor a kölcsönös függetlenség alatt azt fogjuk érteni, hogy ha a végtelen aggregátumokat kicserélnénk véges aggregátumokra, akkor azok kölcsönösen függetlenek lennének. Valószínűségi változók végtelen sok aggregátuma kölcsönösen független ha $A_1,A_2,\ldots$ véges darabszámú aggregátum kölcsönösen függetlenül kiválasztható ezek közül. 

Az itt megjelent definíciók komplex valószínűségi változókra is igazak ha (\ref{eq:kfvv})-ben megengedjük, hogy $X_j$ a két dimenziós komplex sík Borel halmaza legyen és ez a definíció később majd hasznosnak is fog bizonyulni. Az előzőeket komplex valószínűségi változókra természetesen úgy is lehet érteni, miszerint azok valós valószínűségi változók párok, egy valós és egy képzetes résszel. 

Most vegyünk egy $x$ valószínűségi változót, és integráljuk azon az $\omega$ téren, amit $\Omega$-val jelöltünk. Ha ez az integrál létezik, akkor ezt $\mathrm{\textbf{E}}\{x\}$-nek fogjuk jelölni (bizonyos esetekben $\mathrm{\textbf{E}}\{x(\omega)\}$-nek) és ezt az $x$ valószínűségi változó \textit{várható értékének} hívjuk majd.
\begin{equation}
 \mathrm{\textbf{E}}\{x\} = \int_{\Omega} x d \mathrm{\textbf{P}}.
\end{equation}
Tudjuk, hogy ez az integrál csak akkor létezik, ha $|x|$ integrálja véges. 

\subsection{A konvergenciáról}

Vegyünk $x,x_1,x_2,\ldots$ valószínűségi változókat. Ekkor ha teljesül az, hogy majdnem minden $\omega$-ra: 
\begin{equation}
 \lim_{n \to \infty} x_{n}(\omega) = x(\omega),
\end{equation}
akkor azt mondjuk, hogy:
\begin{equation}
 \lim_{n \to \infty} x_{n} = x.
\end{equation}
$1$ valószínűséggel teljesül. Ilyen jellegű konvergenciáról csak akkor beszélhetünk, ha  
\begin{equation}
	\lim_{n \to \infty} \mathrm{\textbf{P}}\{\underset{m \geq n}{\mathrm{L.F.K.}} |x_{m}(\omega) - x_{n}(\omega) | \geq \epsilon \} = 0,
\end{equation}
és ennek minden pozitív $\epsilon$-ra teljesülni kell. Itt a legkisebb felső korlátra bevezettük az L.F.K. rövidítést. A \textit{sztochasztikus konvergenica} egy ennél gyengébb feltétel:
\begin{equation}
	\lim_{n \to \infty} \mathrm{\textbf{P}}\{|x_{n}(\omega) - x(\omega) | \geq \epsilon \} = 0.
\end{equation}
Ennek minden pozitív $\epsilon$-ra teljesülni kell. A sztochasztikus konvergenicát így jelöljük: 
\begin{equation}
	\underset{n \to \infty}{\operatorname{plim}} x_{n} = x.
\end{equation}
Erre gyakran \textit{valószínűségben vett konvergenica}-ként és \textit{mértékben vett konvergenica}-ként hivatkoznak. Az $1$ valószínűséggel teljesülő konvergenica és a valószínűségben vett konvergenica kapcsolata jól ismert, itt most nem fogjuk bebebizonyítani a következőeket:
\begin{enumerate}[label=(\alph*)]
 \item az $1$ valószínűséggel teljesülő konvergenica magával vonja a valószínűségben vett konvergenciát
 \item $\underset{n \to \infty}{\operatorname{plim}} x_{n} = x$ csak akkor teljesül ha az $x_j$ sor minden $x_{a_n}$ részsora tartalmaz olyan részsort, amely $x$-hez $1$ valószínűséggel konvergál.
\end{enumerate}
Azt mondjuk továbbá, hogy egy $\{x_n\}$ sor $x$-hez \textit{középértékben} konvergál ha $\mathrm{\textbf{E}}\{|x_n|^2\} < \infty$ minden $n$-re és ha $\mathrm{\textbf{E}}\{|x|^2\} < \infty$ és ha
\begin{equation}
\lim_{n \to \infty} \mathrm{\textbf{E}}\{|x-x_n|^2\} = 0.
\end{equation}
Erre a következő jelölést használjuk:
\begin{equation}
\underset{n \to \infty}{\operatorname{l.i.m.}} x_{n} = x.
	\label{eq:kek}
\end{equation}
A középértékben vett konvergencia magával vonja a valószínűségben vett konvergeniciát. Valójában ha (\ref{eq:kek}) teljesül, akkor minden $\epsilon > 0$-ra teljesülni fog, hogy 
\begin{equation}
	\lim_{n \to \infty} \mathrm{\textbf{P}}\{|x_{n}(\omega) - x(\omega) | \geq \epsilon \} \leq \lim_{n \to \infty} \frac{\mathrm{\textbf{E}}\{|x_n-x|^2\}}{\epsilon^2} = 0.
\label{eq:kek2}
\end{equation}
Most vegyünk egy egydimenziós eloszlásfüggvény sort és ezt jelöljük $\{F_n\}$-el. A legcélszerűbb ha az eloszlásfüggvényhez való konvergenciát úgy definiáljuk, hogy $\lim_{n \to \infty} F_{n}(\lambda) = F(\lambda)$ teljesül $F$ minden folytonossági pontjában. Ha ez a feltétel teljesül, akkor akkor a konvergencia egyenletes lesz $F$ minden zárt folytonos intervallumán mindegy hogy az véges-e vagy végtelen. Hogyan tudunk most két eloszlásfüggvény --~$G_1$ és $G_2$ között~-- olyan távolság-definíciót bevezetni, ami ilyen jellegű konvergenciát eredményez? Ehhez rajzoljuk fel $G_1$-et és $G_2$-t, majd a szakadási pontokba húzzunk be függőleges vonalakat és ekkor a két függvény között a távolságot definiáljuk úgy, hogy maximálisan mennyit kell megtenni a két függvény között egy $-1$ meredekségű egyenes mentén. A távolság ilyen definíciója mellett az eloszlásfüggvények tere teljes metrikus tér lesz és $\lim_{n \to \infty} F_{n}(\lambda) = F(\lambda)$ teljesül $F$ minden folytonossági pontjában, de csak akkor ha $F_n$ és $F$ között a távolság $0$-hoz tart $1/n$-el.

Azt mondjuk továbbá, hogy az $x$ valószínűségi változó \textit{eloszlásban konvergál} ha az $\{F_n\}$ függvénysora az $\{x_n\}$ valószínűségi változóknak az előbb ismertetett módon konvergál egy eloszlásfüggvényhez. Ez a meghatározás valamelyest szerencsétlen, hiszen lesznek olyan esetek, amikor $\{x_n\}$ sehogy sem konvergál. Ha például az eloszlásfüggvények megegyeznek, de ettől függetlenül a valószínűségi változókat tetszőlegesen megválaszthatjuk (mondjuk kölcsönösen függetlenre választjuk őket) és így eloszlásban konvergáló valószínűségi változó sort kapunk.

\subsection{A valószínűségi változók családjai}

Vannak olyan esetek, amikor a valószínűségi változókat közvetlen módon adjuk meg. Ha mondjuk vesszük a számegyenest, mint az $\omega$ teret, és az $\omega$ halmazoknak a Lebesgue mérhető halmazokat választjuk meg és ha a valószínűségi mértéket így definiáljuk:
\begin{equation}
 \mathrm{\textbf{P}}\{ A \} = \frac{1}{\sqrt{2\pi}} \int_{A} e^{-\lambda^2/2}d\lambda,
\end{equation}
akkor az $1,\omega,\omega^2,\ldots$ függvénysor valószínűségi változók sora lesz. Sokszor azonban teljesen mindegy, hogy milyen $\omega$ téren dolgozunk és csak azt követeljük meg, hogy létezzen a valószínűségi változók egy olyan osztálya, ami kielégít bizonyos feltételeket. Ebben az esetben az történik, hogy véges aggregátumok többváltozós eloszlásai vannak megadva és feltesszük, hogy létezik a valószínűségi változók olyan családja, amire az adott eloszlás lesz igaz. Most ezt a gondolatot fogjuk kicsit részletesebben kifejteni. \ref{sec:veletlenfolyamatokdefinicioja}§2-ben lesz még szó arról hogyan kell más módokon megadni eloszlásfüggvényeket.

Például, amikor majd így kezdünk egy tételt: "vegyünk $x_1,\ldots,x_n$ kölcsönösen független valószínűségi változókat, amelyek eloszlásfüggvényei $F_1,\ldots,F_n$" a tétel triviális lesz anélkül, hogy kimondanánk egy tételt az $n$ valószínűségi változóról valamilyen téren. Nézzünk most konkrétan egy ilyen létezésről szóló tételt. Ez remélhetőleg tisztává fogja varázsolni a képet. Vegyük az $\omega$ teret, amiben $n$ dimenzióban $\xi_1,\ldots,\xi_n$ pontok vannak. A Borel halmazokon az $\omega$ téren a következő módon lehet valószínűségi mértéket definiálni: 
\begin{equation}
	\mathrm{\textbf{P}}\{ A \} = \int\limits_{\substack{A}} \ldots \int\limits_{\substack{A}}  dF_1(\xi_1) \ldots dF_n(\xi_n),
\end{equation}
ahogyan erről már §\ref{ss:vveelo}-ben volt szó. Most pedig jelöljük $x_j$-vel a $j$. koordináta változót, avagy $x_j(\omega)=\xi_j$ ha $\omega$ a ($\xi_1,\ldots,\xi_n$) pont. Ebben az esetben az $x_j$ valószínűségi változók kölcsönösen függetlenek, eloszlásfüggvényük pedig $F_1,\ldots,F_n$. Ezáltal tehát bebizonyítottuk, hogy találhatunk olyan $\xi_1,\ldots,\xi_n$ valószínűségi változókat, amelyek kielégítik a kívánt feltételeket és hogy pontosan ezeket a valószínűségi változókat lehet úgy értelmezni, mint koordináta-függvényeket az $n$ dimenziós térben. Most pedig nézzünk egy általánosított eljárást, ami valószínűségi változók teljesen általános aggregátumaira is érvényes lesz. 

Általánosságban meg fogunk adni egy index osztályt és ezt $T$-vel fogjuk jelölni, majd ehhez valószínűségi változók osztályát definiáljuk. Ezt jelöljük $\{x_t,t \in T\}$-vel. Ha veszünk egy tetszőleges véges $t$ halmazt ($t_1,\ldots,t_n$) akkor $\xi_t,\ldots,\xi_{t_n}$ többváltozós eloszlása ($F_{t_1,\ldots,t_n}$) meghatározott. Az így meghatározott többváltozós eloszlásfüggvénynek kölcsönösen konzisztensnek kell lennie. Ez alatt azt értjük, hogy ha permutáljuk az indexeket, $\alpha_1,\ldots,\alpha_n$ szerint, akkor teljesülnie kell annak, hogy:
\[
	F_{t_1,\ldots,t_n} (\lambda_1,\ldots,\lambda_n) \equiv F_{t_{\alpha_1},\ldots,t_{\alpha_n}} (\lambda_{\alpha_1},\ldots,\lambda_{\alpha_n}),
\]
és ha $m < n$:
\[
	F_{t_1,\ldots,t_m} (\lambda_1,\ldots,\lambda_m) \equiv \lim_{\substack{\lambda_j \to \infty \\ j=m+1,\ldots,n }}F_{t_1,\ldots,t_n} (\lambda_1,\ldots,\lambda_n).
\]
Kolmogorov látta be, hogy csak ezeket a konzisztencia kritériumokat kell megkövetelni. Az $x_t$ valószínűségi változókat a következő módon lehet megadni: Az $\Omega$ térnek pontokból álló teret választunk. $\omega : \xi_t, t \in T$, ahol $\xi_t$ egy tetszőleges valós szám lehet. Az $\Omega$ tér tehát $t \in T$-k függvényéből álló tér. Más megközelítésben, $\Omega$ koordináta tér, dimenziója pedig $T$ kardinalitása. $t=s$-ben a $t$ függvény értéke egy $\omega$ függvényt definiál, ezt $x_s$-el jelöljük ha $x_s(\omega) = \xi_s$, ahol az $\omega$ függvény $\xi_{(\cdot)}$. A következő $\omega$ halmazhoz:
\[
 \{x_{t_j}(\omega) \leq \lambda_j , j=1,\ldots,n\}
\]
a következő számot rendeljük mint mérték:
\[
	F_{t_1,\ldots,t_n} (\lambda_1,\ldots,\lambda_n).
\]
Vagy még általánosabban, a következő $\omega$ halmazhoz:
\[
	\{[x_{t_1}(\omega),\ldots,x_{t_n}(\omega)] \in A\},
\]
ha $A$ $n$ dimenziós Borel halmaz, akkor a következő számot rendeljük mint mérték:
\[
 \int\limits_{\substack{A}} \ldots \int\limits_{\substack{A}}  d_{\xi_1,\ldots,\xi_n}F_{t_1,\ldots,t_n}(\xi_1,\ldots,\xi_n).
\]
Ha ilyen módon rendelünk az $\omega$ halmazokhoz mértéket akkor az adott osztályú $\omega$ halmazok által generált Borel mezőn valószínűségi mértéket kapunk. Az $\omega$ függvények családja $\{\xi_t, t \in T\}$ az így kialakított eloszlásoknak megfelelő valószínűségi változók családja lesz. 

Az alap $\omega$ tér ($\Omega$) most $\Omega_T$ volt az előzőekben, ami minden $t \in T$ függvényt magában foglalja és így a valószínűségi változó család az descartesi térben lévő koordináta-függvények családja lett. Azonban még akkor is ha az alap tér nem a descartesi térben lévő valószínűségi változó család ($\{\xi_t, t \in T\}$), akkor is sok esetben lehet ilyennel helyettesíteni és ezt majd lentebb látni is fogjuk. 

Vegyünk most egy $\{\xi_t, t \in T\}$ valószínűségi változó családot. Ebben az esetben rögzített $\omega$-ra az $x_t(\omega)$ függvényérték $t$ egy függvényét adja meg. Az így kapott $t$ függvényeket a család \textit{mintavételi} függvényének fogjuk hívni. Abban a konkrét esetben, amikor $\Omega$ koordináta tér ($\Omega_T$), és $x_t$ a $t$. koordináta függvény, az $\omega$ bázispont és a mintavételi függvény koncepciója egybeesik. Mindenesetre gyakran praktikus lesz olyan kifejezéseket használni, mint mondjuk "majdnem minden mintavételi függvény" és emögött mindig azt a mértékelméleti koncepciót fogjuk érteni ami visszautal az $\Omega$ térre. Például a \textit{majdnem minden mintavételi függvény} kifejezés egyenértékű azzal, hogy "majdnem minden $\omega$". Ha a $T$ paraméterhalmaz véges vagy megszámlálhatóan végtelen, akkor helyesebb \textit{mintavételi sorról} beszélni, mint mintavételi függvényekről és általában így is fogunk eljárni.

Végül még említsük meg azt is, hogy amikor eloszlásfüggvények osztályáról beszélünk (mint ahogyan a bevezetésben ezt meg is tettük), akkor és ennek segítségével $\Omega_T$ halmazokhoz mértéket definiálunk, gyakran célszerűbb nem az összes $t$ ($t \in T$) függvényosztályt venni, hanem jobban járhatunk ha kikötjük, hogy a $t$ paraméterek csak valamilyen $X$ halmazból kerülhetnek ki. Nincs azonban feltétlenül szükség arra, hogy ezt az $X$ halmazt $t$-től függően állapítsuk meg. Minden eddig fentebb elmondot teljesül akkor is ha $X$ a végtelen számegyenes ($-\infty \geq x \geq \infty$) és a megfelelően megválasztott eloszlások $1$ valószínűséggel korlátozzák a valószínűségi változókat $X$-re.



\section{A Véletlen Folyamatok Definíciója}\label{sec:veletlenfolyamatokdefinicioja}
\section{Folyamatok Kölcsönösen Független Valószínűségi Változókkal}\label{sec:folyamatokkolcsonosenfuggetlenvaloszinusegivaltozokkal}
\section{Folyamatok Kölcsönösen Korrelálatlan vagy Ortogonális Valószínűségi Változókkal}\label{sec:folyamatokkolcsonosenkorrelalatlanvagyortogonalisvaloszinusegivaltozokkal}
\section{Diszkrét Paraméterű Markov Folyamatok}\label{sec:diszkretparameterumarkov}
\section{Folytonos Paraméterű Markov Folyamatok}\label{sec:folytonosparameterumarkov}
\section{Martingálok}\label{sec:martingalok}
\section{Független Növekményű Folyamatok}\label{sec:fuggetlennovekmenyufolyamatok}
\section{Ortogonális Növekményű Folyamatok}\label{sec:ortogonalisnovekmenyufolyamatok}
\section{Diszkrét Paraméterű Stacionárius Folyamatok}\label{sec:diszkretparameterustacionariusfolyamatok}
\section{Folytonos Paraméterű Stacionárius Folyamatok}\label{sec:folytonosparameterustacionariusfolyamatok}
\section{Lineáris Legkisebb Négyzetes Predikció --- Stacionárius Tág Értelmű Folyamatok}\label{sec:llsp}


\end{document}

